\documentclass[12pt]{article}
\usepackage{microtype}
\usepackage{marginnote}
\usepackage[top=1.5cm, bottom=1.5cm, outer=2cm, inner=5cm, heightrounded, marginparwidth=2.5cm, marginparsep=1cm]{geometry}

\setlength\parindent{0pt}
\setlength{\parskip}{1em}

\reversemarginpar

\begin{document}

\section*{Emily \& Trevor Wedding Ceremony}

\subsection*{11 am Tuesday, February 17, 2015}

\marginnote{\subsubsection*{\textsl{Welcome}}}
Welcome family, friends and loved ones. My name is Mari Sargent and I
have the privilege of performing this marriage ceremony today. We
gather here today to celebrate the wedding of Trevor and Emily. You
have come here to share in this formal commitment they make to one
another, to offer your love and support to this union, and to allow
Trevor and Emily to start their married life together surrounded by
the people dearest and most important to them. \par

So welcome to one and all, who have traveled from near and far. Trevor
and Emily thank you for your presence here today, and now ask for your
blessing, encouragement, and lifelong support, for their decision to
be married.

\marginnote{\subsubsection*{\textsl{Definition of Marriage}}}
Marriage is perhaps the greatest and most challenging adventure of
human relationships. No ceremony can create your marriage; only you
can do that --- through love and patience; through dedication and
perseverance; through talking and listening, helping and supporting,
and believing in each other; through tenderness and laughter; through
learning to forgive, learning to appreciate your differences, and by
learning to make the important things matter, and to let go of the
rest. What this ceremony can do is to witness and affirm the choice
you make to stand together as lifemates and partners.

\marginnote{\subsubsection*{\textsl{Declaration of Intent}}}
Will you, Trevor, take this woman to be your wedded wife? \par

\textsl{(Trevor: I will.)} \par

Will you, Emily, take this man to be your wedded husband? \par

\textsl{(Emily: I will.)} \par

\marginnote{\subsubsection*{\textsl{Readings}}}
In the spirit of the importance of family to a marriage, Trevor and
Emily have asked Trevor's brother Jason and Emily's sister Alex to
read selections about love and partnership that especially resonate
with them. We will begin with Jason, who will read a quote from Bob
Marley. \par

\textit{(Jason: He's not perfect. You aren't either, and the two of
  you will never be perfect. But if he can make you laugh at least
  once, causes you to think twice, and if he admits to being human and
  making mistakes, hold onto him and give him the most you can. He
  isn't going to quote poetry, he's not thinking about you every
  moment, but he will give you a part of him that he knows you could
  break. Don't hurt him, don't change him, and don't expect for more
  than he can give. Don't analyze. Smile when he makes you happy, yell
  when he makes you mad, and miss him when he's not there. Love hard
  when there is love to be had. Because perfect guys don't exist, but
  there's always one guy that is perfect for you.)} \par

Thank you, Jason. Now we have Alex to do a reading from ``I Like You''
by Sandol Stoddard Warburg. \par

\textsl{(Alex: \\
  I like you and I know why. \\
  I like you because you are a good person to like. \\
  I like you because when I tell you something special,
  you know it's special \\
  And you remember it a long, long time. \\
  You say, "`Remember when you told me something special?"' \\
  And both of us remember} \par

\textsl{When I think something is important \\
  you think it's important too \\
  We have good ideas \\
  When I say something funny, you laugh \\
  I think I'm funny and you think I'm funny too \\
  Hah-hah!} \par

\textsl{...And I like you because when I am feeling sad \\
  You don't always cheer me up right away \\
  Sometimes it is better to be sad... \\
  I like you because if I am mad at you \\
  Then you are mad at me too \\
  It's awful when the other person isn't...} \par

\textsl{I like you because I don't know why but \\
  Everything that happens is nicer with you \\
  I can't remember when I didn't like you \\
  It must have been lonesome then \\
  I like you because because because \\
  I forget why I like you but I do.) }\par

Thank you, Alex.

\marginnote{\subsubsection*{\textsl{Support of Community}}}
Two people in love do not live in isolation. Their love is a source of
strength with which they may nourish not only each other but also the
world around them. And in turn, we, their community of friends and
family, have a responsibility to this couple. By our steadfast care,
respect, and love, we can support their marriage and the new family
they are creating today. \par

Will everyone please rise. \par

Will you who are present here today surround Trevor and Emily in love,
offering them the joys of your friendship, and supporting them in
their marriage? \par

\textsl{(All: We will.)} \par

You may be seated.

\marginnote{\subsubsection*{\textsl{Wedding Vows}}}
We have come to the point of your ceremony where you are going to say
your vows to one another. But before you do that, I ask you to
remember that love---which is rooted in faith, trust, and
acceptance---will be the foundation of an abiding and deepening
relationship. No other ties are more tender, no other vows more sacred
than those you now assume. If you are able to keep the vows you take
here today, not because of any religious or civic law, but out of a
desire to love and be loved by another person fully, without
limitation, then your life will have joy and the home you establish
will be a place in which you both will find the direction of your
growth, your freedom, and your responsibility. \par

Trevor and Emily have chosen to write their own vows. Trevor, please
read the vows you have written for Emily. \par

\textsl{(Trevor: wedding vows.)} \par

Emily, please read the vows you have written for Trevor. \par

\textsl{(Emily: wedding vows.)} \par

The inward and invisible bond that you both share, as apparent from
your vows, is shown outwardly and visibly through your wedding
rings. Neil, may I have the rings, please? \par

Trevor, please repeat after me: I give you this ring, as a daily
reminder of my love for you. \par

\textsl{(Trevor: I give you this ring, as a daily reminder of my love
  for you.)} \par

Emily, please repeat after me: I give you this ring, as a daily
reminder of my love for you.) \par

\textsl{(Emily: I give you this ring, as a daily reminder of my love
  for you.)} \par

\marginnote{\subsubsection*{\textsl{Signing of the register}}}
While this wedding is a celebration of love and commitment, it is also
a legal ceremony. \par

The signing of a marriage certificate is not just a legal
requirement. It is an ancient custom that represents the concept that
not only must marriage be entered into consensually by both parties,
but that it is also a social contract between a couple and their
community, symbolized by witnesses, for the good of all. So, I ask
Emily's sister Michelle and Trevor's brother Jason to join me and the
bride and groom at the table to sign the documents before them.

\marginnote{\subsubsection*{\textsl{Introduction of the couple}}}
By the power of your love and commitment, and the power vested in me,
I now pronounce you husband and wife! You may kiss each other! \par

It is my personal privilege and a great joy to be the first one to
introduce Trevor and Emily as husband and wife. Please greet them
warmly. \par

\textsl{(Pete (photographer): \textsl{Gathers guests and couple
    together for pictures})}

\end{document}
